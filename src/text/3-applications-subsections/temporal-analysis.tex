\subsection{Temporal analysis}
\label{ssec:temporal-analysis}

A common use case in most astrophysical scenarios is to study the temporal
variability of the source. The most basic way to do this is to construct a
lightcurve, ie, the flux of the source in each given time bin. In gammapy, this
is done by using the LightCurveEstimator which fits the normalisation of the
source in each energy band per observation, keeping other parameters constant.
For custom time binning, an observation needs to be split into finer ones using
the \verb"Observation.select_time method". Figure \ref{fig:hess_lightcurve_pks}
shows the lightcurve of the blazar PKS~2155-304 in different energy bands as
observed by the H.E.S.S. telescope during an exceptional flare on the night of
July 29 - 30, 2006~\cite{2009A&A...502..749A}. The data is available publicly
as a part of the HESS-DL3-DR1~\cite{HESS-DL3-DR1}, and shipped with
\verb"GAMMAPY_DATA". Each observation is first split into 10 min smaller
observations, and spectra extracted for each of these within a 0.11 deg radius
around the source. A PowerLawSpectralModel is fit to all the datasets, leading
to a reconstructed index of $3.54 \pm 0.02$. With this assumed spectral model
the \verb"LightCurveEstimator" run directly for two energy bands, 0.5 - 1.5
TeV, and 1.5 - 20 TeV,	respectively.

\begin{figure*}[t]
	\centering
	\includegraphics[width=1.\textwidth]{figures/hess_lightcurve_pks.pdf}
	\caption{10 min binned lightcurve for PKS~2155-304 in two energy bands, (500
		GeV - 1.5 TeV, and 1.5 TeV to 20 TeV) as observed by the H.E.S.S. telescopes in
		2006.} \label{fig:hess_lightcurve_pks} \end{figure*}

The obtained flux points can be analytically modelled using the available, or
user-implemented temporal models. Alternatively, instead of  extracting a
lightcurve, it is also possible to directly fit temporal models to the reduced
datasets. By associating an appropriate SkyModel, consisting of both temporal
and spectral components, or using custom temporal models with spectroscopic
variability, to each dataset, a joint fitting across the datasets will directly
return the best fit temporal and spectral parameters.
